\documentclass[numbers=noenddot,12pt,a4paper]{scrartcl}
\usepackage[greek,ngerman]{babel}
\usepackage[T1]{fontenc}
\usepackage[utf8]{inputenc}
\usepackage{fullpage}
\usepackage{libertine}
\usepackage{ziffer}
\usepackage{graphicx}
\usepackage{units}
\usepackage[infoshow]{tabularx}
\usepackage{amsmath}
\usepackage{amssymb}
\usepackage{wrapfig}
\usepackage{esint}
\usepackage{float}
\usepackage{wrapfig}
\usepackage[font=small]{caption}
\usepackage{subcaption}
\usepackage{lscape}
\usepackage{hyperref}

\renewcommand{\thefigure}{Abb. \arabic{figure}}

\captionsetup[wrapfigure]{name=}
\captionsetup[figure]{name=}
\newcommand{\degree}{^\circ}
\newcommand{\diff}{\textnormal{d}}
\newcommand{\tenpo}[1]{\cdot 10^{#1}}
\newcommand{\greek}[1]{\greektext#1\latintext}
\newcommand{\ix}[1]{_\text{#1}}
\newcommand{\imag}{\mathbf{i}}
\newcommand{\tilt}[1]{\textit{#1}}
\newcommand{\grad}[1]{\textit{grad}\left(#1\right)}
\newcommand{\divergenz}[1]{\textit{div}\left(#1\right)}
\newcommand{\euler}{\mathnormal{e}}
\newcommand{\fett}[1]{\textbf{#1}}
\newcommand{\partiell}[2]{\frac{\partial #1}{\partial #2}}
\newcommand{\partiellz}[2]{\frac{\partial^2 #1}{\partial #2^2}}
\newcommand{\partielld}[2]{\frac{\partial^3 #1}{\partial #2^3}}

\title{Protokoll: Akustische Experimente zur Modellierung von Quanten-Phänomenen}
\author{Tom Kranz, Philipp Hacker}
\date{\today}

\begin{document}
\maketitle
\begin{center}
Betreuer: S. Peglow\\
Versuchsdatum: 11.11.2014\\
\begin{table}[h]
\centering
Note: %TODO Gute Note erhalten :)
\begin{tabularx}{1.5cm}{|X|}
\hline \\ \\
\hline
\end{tabularx}
\end{table}
\end{center}
\vspace*{\fill}
\tableofcontents
\vfill
\newpage
\section{Einleitung}
In diesem Versuch sollen die Eigenschaften grundlegender quantenmechanischer Systeme anhand einfacher Experimente mit Schall veranschaulicht werden. Dazu werden Ähnlichkeiten der quantenmechanischen Bewegungsgleichung, der Schrödinger-Gleichung, und der Bewegungsgleichung für klassische Wellenerscheinungen, der Wellengleichung, unter bestimmten Randbedingungen ausgenutzt, aber auch auf die Unterschiede aufmerksam gemacht.\\
Die durchgeführten Experimente modellieren ein Teilchen im Kastenpotential, ein Elektron im Coulomb-Feld eines Atomkerns, sowohl ungestört, als auch unter Einwirkung eines symmetriebrechenden äußeren Feldes und ein Elektron im $\text{H}_2^+$-Molekül bei unterschiedlich starker Kopplung.
\section{Grundlagen}
In der Quantenmechanik werden Zustände durch sogenannte Wellenfunktionen beschrieben. Sie sind Lösungen der Schrödinger-Gleichung für die Randbedingungen, denen das betrachtete physikalische Objekt unterliegt. In Ortsdarstellung hat sie im Allgemeinen die Form:
\begin{align}
\imag\hbar\frac{\partial\psi}{\partial t}=-\frac{\hbar^2}{2m}\vec{\nabla}^2 \psi+V(\vec{r})\psi\label{eq:schro}
\end{align}
Mit der Wellenfunktion $\psi(\vec{r},t)$, dem reduzierten Planckschen Wirkungsquantum $\hbar$, der Teilchenmasse $m$, dem Laplace-Operator $\vec{\nabla}^2=\overset{3}{\underset{i=1}{\sum}}\frac{\partial^2}{\partial r_i^2}$ und dem ortsabhängigen Potential $V(\vec{r})$.\\
Die klassische Wellengleichung hat im Allgemeinen die Form:
\begin{align}
\frac{1}{c^2}\frac{\partial^2 u}{\partial t^2}=\vec{\nabla}^2 u
\end{align}
Hier ist $u(\vec{r},t)$ eine physikalische Größe (zum Beispiel der Druck $p$) und $c$ die Ausbreitungsgeschwindigkeit der Welle.\\
Sofort fällt die Ähnlichkeit zwischen den beiden Gleichungen auf: Die linke Seite ist eine Ableitung nach der Zeit, die rechte enthält einen Orts-Differentialoperator. Jedoch ist die Zeitableitung in der Wellengleichung eine zweifache, in der Schrödinger-Gleichung eine einfache, aber mit der imaginären Einheit als Vorfaktor. Auch tritt in der Wellengleichung kein Potential auf.
\subsection{Das Teilchen im Kastenpotential}
Das Teilchen im Kastenpotential ist ein Modell eines Teilchens, das sich in einem schwachen Potential bewegt, das abrupt von hohen Potentialwänden umgeben ist. Das (eindimensionale) Modell vereinfacht dies auf das Potential
\begin{align}
V(x)=\left\{\begin{array}{ll}
0&,\,0< x< L \\ 
\infty &\text{, ansonsten}
\end{array} \right.
\end{align}
Wobei $L$ die Länge des Kastens darstellt. Außerhalb des Kastens können keine Zustände existieren, weswegen dort die Wellenfunktion $\psi=0$ ist; innerhalb des Kastens vereinfacht sich die Schrödingergleichung zu
\begin{align}
\imag\hbar\frac{\partial\psi}{\partial t}=-\frac{\hbar^2}{2m}\frac{\partial^2\psi}{\partial x^2}
\end{align}
Diese wird durch eine Linearkombination von Eigenzuständen der stationären Schrödingergleichung, multipliziert mit einem Phasenfaktor $\exp\left(-\imag\omega t\right)$, gelöst. Die stationäre Schrödingergleichung ist die Eigenwertgleichung der rechten Seite der Gleichung (\ref{eq:schro}), in diesem Falle:
\begin{align}
E\psi=-\frac{\hbar^2}{2m}\frac{\partial^2\psi}{\partial x^2}\label{eq:kasten}
\end{align}
Die Lösungen dieser Gleichung sind, unter Beachtung der Randbedingungen, Sinus-Funktionen
\begin{align}
\psi(x)=A\sin\left(kx\right)\text{ mit }k=n\frac{\pi}{L};\;n\in\mathbb{N}\label{eq:kastenl}
\end{align}
Über die aus Gleichung (\ref{eq:kasten}) folgende Dispersionsrelation $E=\frac{\hbar^2k^2}{2m}$ bewirken die Randbedingungen auch eine Quantelung der Energie der Eigenzustände.
\subsubsection{Analogon: Stehende Schallwellen in der Röhre}
Werden Schallwellen in einem Rohr angeregt, werden sie von den Enden des Rohres reflektiert und alle im Rohr befindlichen Wellen überlagern sich -- stehende Wellen entstehen. Wenn sich alle im Rohr befindlichen Schallwellen konstruktiv überlagern, spricht man von Resonanz. Für die konstruktive Überlagerung von einfallender und reflektierter Welle muss die Wellenlänge $\lambda$ der Schallwelle ein halbzahliges Vielfaches der Rohrlänge $L$ sein:
\begin{align}
\frac{n}{2}\lambda=L
\end{align}
Schaut man zurück auf Gleichung (\ref{eq:kastenl}), fällt auf, dass bei Identifizierung von $k=\frac{2\pi}{\lambda}$ genau der gleiche Ausdruck entsteht. Resonanzen von Schallwellen in einem Rohr lassen sich also als einfache Modellierung der Eigenfunktionen des Teilchens im Kasten ansehen. Allerdings findet hier die Energie, mit ihrer quadratischen Abhängigkeit von der Quantenzahl $n$ keine Entsprechung. Auch sollte beachtet werden, dass die quantenmechanische Wellenfunktion ihre Knotenpunkte an den Rändern hat, während der Druck bei der Resonanz, bedingt durch die Position der Quelle der Erregung, Wellenberge an den Rändern aufweist.
\subsection{Das Wasserstoffatom}
Das Wasserstoffatom ist das einfachste Atom; es besteht lediglich aus einem Proton und einem Elektron. Bei diesem Problem tritt das Coulomb-Potential $V(\vec{r})=V(r)=\frac{-e^2}{4\pi\varepsilon_0 r}$ auf, die stationäre Schrödinger-Gleichung für die Wellenfunktion des Elektrons lautet daher:
\begin{align}
E\psi=-\frac{\hbar^2}{2m}\vec{\nabla}^2 \psi-\frac{e^2}{4\pi\varepsilon_0 r}\psi\label{eq:wass}
\end{align}
Schreibt man diese in Kugelkoordinaten um:
\begin{align}
E\psi=\frac{\hbar^2}{2mr^2}\partiell{}{r}\left(r^2\partiell{\psi}{r}\right)+\frac{\hbar^2}{2mr^2\sin(\theta)}\partiell{}{\theta}\left(\sin(\theta)\partiell{\psi}{\theta}\right)+\frac{\hbar^2}{2mr^2\sin^2(\theta)}\partiellz{\psi}{\varphi}-\frac{e^2}{4\pi\varepsilon_0 r}\psi,\label{eq:wassku}
\end{align}
so eröffnet sich ein Separationsansatz: $\psi(r,\theta,\varphi)=Y^m_l(\theta,\varphi)\chi_l(r)$. Hierbei sind die Kugelflächenfunktionen $Y_l^m$ die Lösungen des Winkelanteils von Gleichung (\ref{eq:wassku}):
\begin{align}
-\frac{1}{\sin(\theta)}\partiell{}{\theta}\left(\sin(\theta)\partiell{Y_l^m}{\theta}\right)-\frac{1}{\sin^2(\theta)}\partiellz{Y_l^m}{\varphi}=l(l+1)Y_l^m
\end{align}
und $\chi_l(r)$ die Lösungen des radialen Anteils von Gleichung (\ref{eq:wassku}):
\begin{align}
-\frac{\hbar^2}{2mr}\partiellz{}{r}\left(r\chi_l\right)-\frac{l(l+1)\hbar^2}{2mr^2}\chi_l--\frac{e^2}{4\pi\varepsilon_0 r}\chi_l=E\chi_l
\end{align}
\subsubsection{Analogon: sphärischer akustischer Resonator}
Werden Schallwellen in einer Hohlkugel angeregt, folgt der Druck innerhalb der Kugel der Wellengleichung
\begin{align}
\partiellz{p}{t}=\frac{1}{\rho\kappa}\vec{\nabla}^2p
\end{align}
Hier sind $\rho$ die Luftdichte und $\kappa$ die Kompressibilität. Ähnlich, wie die stationäre Schrödingergleichung auf einem Separationsansatz beruht, lässt sich diese Gleichung durch den Ansatz $p(\vec{r},t)=p(\vec{r})\cos(\omega t)$ in eine stationäre Gleichung, die Helmholtzgleichung, überführen:
\begin{align}
\omega^2p(\vec{r})=-\frac{1}{\rho\kappa}\vec{\nabla}^2p(\vec{r})\;\text{ bzw. }\;-\frac{\omega^2}{c^2}p=\vec{\nabla}^2p
\end{align}
Wie Gleichung (\ref{eq:wassku}) bringt auch hier das Umschreiben in Kugelkoordinaten die Möglichkeit der Separation $p(r,\theta,\varphi)=Y_l^m(\theta,\varphi)f(r)$ mit sich:
\begin{align}
-\frac{1}{r^2}\partiell{}{r}\left(r^2\partiell{p}{r}\right)-\frac{1}{r^2\sin(\theta)}\partiell{}{\theta}\left(\sin(\theta)\partiell{p}{\theta}\right)-\frac{1}{r^2\sin^2(\theta)}\partiellz{p}{\varphi}=\frac{\omega^2}{c^2}p\label{eq:sph}
\end{align}
Der Winkelanteil der Gleichung ist:
\begin{align}
-\frac{1}{\sin(\theta)}\partiell{}{\theta}\left(\sin(\theta)\partiell{Y_l^m}{\theta}\right)-\frac{1}{\sin^2(\theta)}\partiellz{Y_l^m}{\varphi}=l(l+1)Y_l^m,
\end{align}
wohingegen der radiale Anteil wie folgt aussieht:
\begin{align}
-\partiellz{f}{r}-\frac{2}{r}\partiell{f}{r}+\frac{l(l+1)}{r^2}f(r)=\frac{\omega^2}{c^2}f(r)
\end{align}
Sofort fällt auf, dass die Winkelanteile der Gleichungen (\ref{eq:wassku}) und (\ref{eq:sph}) und damit auch der Wellenfunktion, beziehungsweise der Funktion des Drucks, identisch sind. Man kann also mit Schallwellen in einer Hohlkugel die Winkelabhängigkeit der Wellenfunktion des Elektrons im Wasserstoffatom modellieren. Die Unterschiede in den Radialanteilen lassen jedoch keine Analogie zwischen den radialen Abhängigkeiten der Wellenfunktion und des Drucks zu.
\subsubsection{Gebrochene Symmetrie}
Der sphärische Resonator in unserem Versuch besteht aus zwei halben Hohlkugeln, die zusammengesteckt werden, um eine volle Hohlkugel zu bilden. In die obere Hälfte ist ein Mikrofon integriert, in die untere ein Lautsprecher (und ein Mikrofon; dieses wird jedoch nicht genutzt) und beide haben zur Kontaktebene einen Winkel von $\unit[45]{\degree}$. Bei einer perfekten Kugel wird die Symmetrieachse (die $z$-Achse) durch die Position des Lautsprechers bestimmt, weswegen nur diejenigen Resonanzen angeregt werden können, deren Kugelflächenfunktionen bei $\theta=0$ nicht verschwinden.
\section{Auswertung}
\section{Anhang}
\end{document}