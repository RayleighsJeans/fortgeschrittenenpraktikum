\documentclass[numbers=noenddot,12pt,a4paper]{scrartcl}
\usepackage[greek,ngerman]{babel}
\usepackage[T1]{fontenc}
\usepackage[utf8]{inputenc}
\usepackage{fullpage}
\usepackage{libertine}
\usepackage{ziffer}
\usepackage{graphicx}
\usepackage{units}
\usepackage[infoshow]{tabularx}
\usepackage{amsmath}
\usepackage{amssymb}
\usepackage{wrapfig}
\usepackage{esint}
\usepackage{float}
\usepackage{wrapfig}
\usepackage[font=small]{caption}
\usepackage{subcaption}
\usepackage{lscape}
\usepackage{hyperref}

\renewcommand{\thefigure}{Abb. \arabic{figure}}

\captionsetup[wrapfigure]{name=}
\captionsetup[figure]{name=}
\newcommand{\degree}{^\circ}
\newcommand{\diff}{\textnormal{d}}
\newcommand{\tenpo}[1]{\cdot 10^{#1}}
\newcommand{\greek}[1]{\greektext#1\latintext}
\newcommand{\ix}[1]{_\text{#1}}
\newcommand{\imag}{\mathbf{i}}
\newcommand{\tilt}[1]{\mbox{\textit{#1}}}
\newcommand{\grad}[1]{\textit{grad}\left(#1\right)}
\newcommand{\divergenz}[1]{\textit{div}\left(#1\right)}
\newcommand{\euler}{\mathnormal{e}}
\newcommand{\fett}[1]{\textbf{#1}}

\title{Protokoll: Neodym-dotieter YAG-Laser} %TODO Name des Versuchs eintragen
\author{Tom Kranz, \underline{Philipp Hacker}} %TODO Protokollschreiber unterstreichen
\date{\today}

\begin{document}
%\setcounter{page}{2}
%\setcounter{section}{1}
\maketitle
\begin{center}
Betreuer: U. Martens\\ %TODO Name des Betreuers eintragen
Versuchsdatum: 25.11.2014\\ %TODO Datum des Versuchs eintragen
\begin{table}[h]
\centering
Note: %TODO Gute Note erhalten :)
\begin{tabularx}{1.5cm}{|X|}
\hline \\ \\
\hline
\end{tabularx}
\end{table}
\end{center}
\vspace*{\fill}
\tableofcontents
\vfill
\newpage
\section{Einleitung}
1964 wurde in den Bell Laboratories (New Jersey, USA) von \tilt{L. Van Uitert} und \tilt{J. E. Geusic} ein Festkörperlaser auf Grundlage eines Neodym-dotierten Yttrium-Aluminium-Granat-Kristalls entwickelt. Dieser hatte den Vorteil, dass er bei relativen hohen Ausgangsleistungen von einigen hundert $\unit{MW}$ im gepulsten Betrieb einen Laserstrahl mit einer geringen Wellenlänge (hauptsächlich $\unit[1064]{nm}$) aussendet. Weitere günstige Charakteristika dieses Klasse 4 Lasers ergeben sich durch die Anregung mit Laserdioden. Diese haben im Infrarotbereich einen Wirkungsgrad von $\sim70 \%$. Für das optische Pumpen des Mediums nutzt man Dioden mit einer Emissionswellenlänge von $\unit[808]{nm}$, was die Effizienz des \tilt{Nd:YAG}-Lasers auf bis zu $50\%$ steigert. Aufgrund der vielen Möglichkeiten, Halbleiterstrukturen zu verändern, kann man die Schwingung nur einer Mode innerhalb der Diode realisieren. Bei geeigneter Stabilisierung der Parameter Temperatur, Pumpstrom und Material lässt sich eine äußerst geringe Frequenzbandbreite des Lasers einstellen. \\
Dieser Versuch konzentriert sich auf die Untersuchungen der Eigenschaften des Laser-Mediums \tilt{Nd:YAG} und der Leistung in Abhängigkeit der Anregung und des Aufbaus. 
\section{Grundlagen}
\section{Auswertung}
\section{Quellen}
%\section{Anhang}
\end{document}