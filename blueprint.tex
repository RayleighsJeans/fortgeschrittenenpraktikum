\documentclass[numbers=noenddot,12pt,a4paper]{scrartcl}
\usepackage[greek,ngerman]{babel}
\usepackage[T1]{fontenc}
\usepackage[utf8]{inputenc}
\usepackage{fullpage}
\usepackage{libertine}
\usepackage{ziffer}
\usepackage{graphicx}
\usepackage{units}
\usepackage[infoshow]{tabularx}
\usepackage{amsmath}
\usepackage{amssymb}
\usepackage{wrapfig}
\usepackage{esint}
\usepackage{float}
\usepackage{wrapfig}
\usepackage[font=small]{caption}
\usepackage{subcaption}

\renewcommand{\thefigure}{Abb. \arabic{figure}}

\captionsetup[wrapfigure]{name=}
\captionsetup[figure]{name=}
\newcommand{\degree}{^\circ}
\newcommand{\diff}{\textnormal{d}}
\newcommand{\tenpo}[1]{\cdot 10^{#1}}
\newcommand{\greek}[1]{\greektext#1\latintext}
\newcommand{\ix}[1]{_\text{#1}}
\newcommand{\imag}{\mathbf{i}}
\newcommand{\tilt}[1]{\textit{#1}}
\newcommand{\grad}[1]{\textit{grad}\left(#1\right)}
\newcommand{\divergenz}[1]{\textit{div}\left(#1\right)}
\newcommand{\euler}{\mathnormal{e}}
\newcommand{\imag}{\mathbf{i}}

\title{Protokoll: Versuch}
\author{Tom Kranz, Philipp Hacker}
\date{\today}

\begin{document}
%\setcounter{page}{2}
%\setcounter{section}{1}
\maketitle
\begin{center}
Betreuer: \\
Versuchsdatum: \\
\begin{table}[h]
\centering
Note:
\begin{tabularx}{1.5cm}{|X|}
\hline \\ \\
\hline
\end{tabularx}
\end{table}
\end{center}
\vspace*{\fill}
\tableofcontents
\vfill
\newpage
\section{Einleitung}
\section{Physikalische Grundlagen}
\section{Durchführung}
\section{Auswertung}
\subsection{Diskussion}
\section{Anhang}
Die originalen Messwert-Aufzeichnungen liegen bei.
\end{document}